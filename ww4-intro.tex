\documentclass[aspectratio=169]{beamer}
\usepackage[T1]{fontenc}
\usepackage[utf8]{inputenc}
\usepackage{listings}
\usepackage{colortbl}

\definecolor{codegreen}{rgb}{0,0.6,0}
\definecolor{codegray}{rgb}{0.5,0.5,0.5}
\definecolor{codepurple}{rgb}{0.58,0,0.82}
\definecolor{backcolour}{rgb}{0.95,0.95,0.92}

\lstdefinestyle{mystyle}{
    backgroundcolor=\color{backcolour},
    commentstyle=\color{codegreen},
    keywordstyle=\color{magenta},
    numberstyle=\tiny\color{codegray},
    stringstyle=\color{codepurple},
    basicstyle=\ttfamily\footnotesize,
    breakatwhitespace=false,
    breaklines=true,
    captionpos=b,
    keepspaces=true,
    numbers=left,
    numbersep=5pt,
    showspaces=false,
    showstringspaces=false,
    showtabs=false,
    tabsize=2
}

\lstset{style=mystyle}

\title{Warewulf\\
making cluster\\
installations fast and \\
reliable}
\date{April 25, Ostrava}
\author{Christian Goll <\texttt{cgoll@suse.com}>}
\usetheme{suse}

\begin{document}

\begin{frame}
\titlepage
\end{frame}
\begin{frame}[fragile]
\frametitle{Introduction}
\framesubtitle{Warewulf is tool for managing beowulf clusters}
\begin{columns}
\column{0.5\textwidth}
\begin{block}{Beowulf}
  \begin{itemize}
    \item old british poem
  \end{itemize}
\end{block}
\begin{block}{Beowulf cluster}
  \begin{itemize}
    \item became popular in the 90.
    \item use of the shelf hardware
    \begin{itemize}
      \item 486 \& linux
      \item \textbf{not} Cray \& unix
    \end{itemize}
    \item warewulf is a typo of werewolf
  \end{itemize}
\end{block}
\column{0.5\textwidth}
  \includegraphics[width=.6\linewidth]{Beowulf}
\end{columns}
\end{frame}

\begin{frame}[fragile]
\frametitle{Introduction}
\framesubtitle{HPC landscape}
\begin{block}{Top five Supercomputers}
\begin{table}
\begin{tabular}{c|l|l|l|l}
1 &	Frontier& EPYC 64C &AMD MI250X &Slingshot-11 \\
2 &	Aurora  & Xeon 9470 & Intel GPU Max&Slingshot-11 \\
3 &	Eagle & Xeon 8480 &  NVIDIA H100 & NVIDIA Infiniband \\
\rowcolor{backcolour}
4 &	Fugaku &  A64FX 48C 2.2GHz & - & Tofu interconnect D \\
5 &	LUMI &  EPYC 64C 2GHz & AMD MI250X & Slingshot-11 
\end{tabular}
\end{table}
\begin{itemize}
  \item only Fugaku uses non standard CPU
  \item others are beowulf clusters with GPUs attached
\end{itemize}
\end{block}
\end{frame}
\begin{frame}[fragile]
\frametitle{Introduction}
\framesubtitle{Beowulf cluster}
\begin{columns}
\column{0.5\textwidth}
\begin{block}{base components}
\begin{itemize}
  \item management node
  \item compute nodes
  \item management network
\end{itemize}
\end{block}
\begin{block}{optional components}
\begin{itemize}
  \item more compute nodes
  \item fast network interconnects 
  \item central storage
  \item bmc/ipmi
\end{itemize}
\end{block}
\column{0.5\textwidth}
  \includegraphics[width=.6\linewidth]{Transtec-056}
\end{columns}
\end{frame}
\begin{frame}[fragile]
\frametitle{Introduction}
\framesubtitle{Beowulf Cluster}
\begin{block}{differences to data centers}
  \begin{itemize}
    \item compute nodes are cattle
    \item hierarchical organization
    \item compute are not updated after boot process
    \item application come from central storage
    \item applications are self compiled
    \item one application can run over several nodes
  \end{itemize}
\end{block}
\end{frame}
%
%\frametitle{Presentation Title}
%\framesubtitle{This is \textit{a subtitle}.}
%
%\begin{columns}
%
%\column{0.5\textwidth}
%This is a text in first column.
%\begin{itemize}
%  \item First item
%  \item Second item
%  \item \texttt{mono spaced font example}
%  \begin{itemize}
%    \item First item
%    \item Second item
%  \end{itemize}
%\end{itemize}
%
%\column{0.5\textwidth}
%This text will be in the second column
%and on a second thought this is a nice looking
%layout in some cases.
%
%\begin{lstlisting}[language=C, caption=C example]
%#include <stdio.h>
%
%/* some comment */
%int main() {
%  printf("Hello World");
%}
%\end{lstlisting}
%
%\end{columns}
%
%\end{frame}
%
%\begin{frame}
%\frametitle{Another slide}
%\framesubtitle{This is \textit{a subtitle}.}
%
%\begin{table}
%  \begin{tabular}{c|c}
%    table1 & trial \\
%    \hline
%    \hline
%    1 & 2 \\
%    3 & 4
%  \end{tabular}
%  \caption*{My table}
%\end{table}
%\end{frame}
%
\end{document}

